%===============================================
%				Ismael Charpentier
%	
%				Formulario MyO:
%	 Temas: Méc. Analitica y Sólido Rígido
%
%===============================================

% Tipo de documento y tamaño letra
\documentclass[14pt]{extarticle}


%===========================================================
%	
%			PACKAGES EMPLEADOS
%
	% ENTRADA Y SALIDA DE DATOS (carácteres)
	%======================================
	\usepackage[T1]{fontenc}
	\usepackage[utf8]{inputenc}
	%======================================


	% FORMATO DEL TEXTO
	%======================================
	\usepackage{indentfirst}
	\usepackage{enumerate}
	%======================================

	%	MATEMATICAS Y FORMULAS
	\usepackage[skins,theorems]{tcolorbox}
	\usepackage{amsmath}
	\usepackage{mathrsfs}
	\usepackage{amssymb}
	\usepackage{xcolor}


	% FORMATO PAGINA
	%=======================================================
	\usepackage[a4paper]{geometry}
	\geometry{left=2.5cm,right=2.5cm,top=2.5cm,bottom=2.5cm}
	%=======================================================
%
%	
%===========================================================

% Formato texto
\setlength{\parindent}{4em}
\setlength{\parskip}{2em}


%===========================================================
%
%					DEFINICION DE MACROS 
% 
% Simbolo Lagrangiana
\newcommand{\Lagr}{\mathscr{L}}
\newcommand{\ddt}{\frac{d}{dt}}
%
%
% 		DEFINICION: Color boxes
% Syntax: \colorboxed[<color model>]{<color specification>}{<math formula>}
\newcommand*{\colorboxed}{}
	\def\colorboxed#1#{%
  		\colorboxedAux{#1}%
	}	
\newcommand*{\colorboxedAux}[3]{%
  % #1: optional argument for color model
  % #2: color specification
  % #3: formula
  	\begingroup
    	\colorlet{cb@saved}{.}%
    	\color#1{#2}%
    	\boxed{%
      		\color{cb@saved}%
      		#3%
   	 	}%
  \endgroup
}
\tcbset{highlight math style={enhanced,
  colframe=red,colback=white,arc=0pt,boxrule=1pt}}
%
%===========================================================


\title{
	\Huge{Formulario Mecánica y Ondas} \\
	\medskip
	\large Sólido Rígido y Mecánica Analítica}

\author{Ismael Charpentier Martín}
\date{2 de Enero, 2021}

\begin{document}

\maketitle
\clearpage

\tableofcontents
\clearpage

\part{Sólido Rígido}

	Un sólido rígido es un cuerpo cuya distancia entre las partículas que lo forman es invariable

	\section{Momento de inercia:}

		\begin{itemize}

			\item{
			Distribución discreta de masa:
			\begin{equation}
				I_{ij} = \sum_\alpha m_\alpha \left( \delta_{ij} \sum_k x^2_{\alpha k} - x_{\alpha i}\hspace*{1mm} x_{\alpha j} \right)
			\end{equation}
			}

			\item{
			Para una distribución continua de masa:
			\begin{equation}
				I_{ij} = \int_V \rho \left( \delta_{ij} \sum_k x^2_{k} - x_{i}\hspace*{1mm} x_{j} \right)
			\end{equation}
			}

			\item{
			En forma matricial:
			\begin{equation}
				\tilde{I} = 
				\begin{pmatrix}
					I_{11} & I_{12} & \cdots & I_{1n}\\
					I_{21} & I_{22} & \cdots & I_{2n}\\
					\vdots & \vdots & \ddots & \vdots \\
					I_{m1} & I_{m2} & \cdots & I_{mn}\\
				\end{pmatrix}
			\end{equation}
			}
		\end{itemize}

		\subsection{Teorema de Steiner:}

			Momento de inercia respecto de un origen distinto al centro de masas (CM), respetando el paralelismo de los ejes:
			\begin{equation}
				I_{ij}\biggr|_o = I_{ij}\biggr| _{cm} + M\left(\delta_{ij}a^2 - a_i \hspace*{1mm} a_j \right)
			\end{equation}

	\section{Energía cinética:}

		Cálculo general de la energía cinética para un sólido rígido:
		\begin{equation}[i]
		 	T = \underbrace{\frac{1}{2} \sum_\alpha m_\alpha v^2_0}_{\text{T. Traslación}} + \underbrace{\frac{1}{2} \sum_\alpha m_\alpha \left( \omega \times r_\alpha \right)^2}_{\text{T. Rotación}} + \underbrace{\frac{1}{2} \sum_\alpha r_\alpha \cdot \left( v_0 \times \omega \right)}_\text{{U.T}}
		\end{equation}

		Este último término es igual a cero cuando ($U.T = 0$):
		\begin{enumerate}[(i)]
			\item{
				$v_0 = 0$: Rotación pura que pasa por O. 
				\begin{equation}
					T = \frac{1}{2} \sum_\alpha m_\alpha \left( \omega \times r_\alpha \right)^2
				\end{equation}
			}
			\item{
				$v_0 \hspace{1mm} || \hspace{1mm} \omega$: Eje de rotación y deslizamientto que pasa por O.
				\begin{equation}
					T = \frac{1}{2} \sum_\alpha m_\alpha v^2_0 + \frac{1}{2} \sum_\alpha m_\alpha \left( \omega \times r_\alpha \right)^2
				\end{equation}
			}
			\item{
				$O \equiv CM$: El origen de coordenada coincide con el centro de masas.
				\begin{equation}
					T = \frac{1}{2} \sum_\alpha m_\alpha v^2_{cm} + \frac{1}{2} \sum_\alpha m_\alpha \left( \omega_{cm} \times r_\alpha \right)^2
				\end{equation}
			}
		\end{enumerate}



\clearpage
\part{Mecánica Analítica:}
	
	\section{Formulación Lagrangiana:}
		\begin{itemize}
		\item{ \textbf{Coordenadas generalizadas:} 
			Se toman tantas como sean necesarias para describir \textit{totalmente} el sistema, es decir, tantas como grados de libertad (g.l)
			\begin{equation}
				q_i \hspace{2cm} i=1,2, \cdots, g.l
			\end{equation}
		}
		\item{ \textbf{Ligaduras:} 
			Fuerza o restricción del movimiento.
				\begin{itemize}
					\item{ Holónomas: Función de la posición de las partículas $(r)$ (puede depender o no explícitamente del tiempo)
						\begin{equation}
							f(r_1, r_2, \cdots, t) = 0
						\end{equation}
					}
					\item{ No holónomas: Además pueden depender de las velocidades: 
						\begin{equation}
							f(r_1, r_2, ..., \frac{dr_1}{dt}, \frac{dr_2}{dt}, ..., t) = 0
						\end{equation}
					}
				\end{itemize}
		}
		\item{ \textbf{Lagrangiana:}
			\begin{equation}
				\tcbhighmath[boxrule=1.5pt]{\Lagr = T - V}
			\end{equation}

			\begin{equation*}
				\begin{split}
					&T \equiv \text{energía cinética del sistema}\\
					&V \equiv \text{potencial del campo}\\
				\end{split}
			\end{equation*}
		}
		\item{ \textbf{Principio de D'Alambert:} 
			\begin{equation}
				\underbrace{\sum_i \left( F^{a}_i - \dot{p}_i \right) \cdot \delta r_i}_{\substack{\text{Trabajo de las fuerzas apli-} \\ \text{cadas más una fuerza inversa}}} + \underbrace{\sum_i \cdot f_i \delta r_i}_{\text{Trabajo fuerzas de ligadura}}
			\end{equation}

			Cuando las fuerzas de ligadura no realizan trabajo (casos vistos en clase)
			\begin{equation}
				\sum_i \left( F^{a}_i - \dot{p}_i \right) \cdot \delta r_i = 0
			\end{equation}
		}
		\item{ \textbf{Ecuaciones de transformación a coord. generalizadas:}

			\begin{itemize}
				\item Posición:			 
					\begin{equation}
							r_i \equiv r_i(q_1, q_2, ..., q_n, t)
					\end{equation}
				\item Desplazamiento virtual: 
					\begin{equation}
							\delta r_i = \sum_j \frac{\partial r_i}{\partial q_j} \cdot \delta q_j
					\end{equation}

				\item Velocidad: 
					\begin{equation}
							v_i \equiv \sum_k \frac{\partial r_i}{\partial q_k} \cdot \dot{q}_k + \frac{\partial r_i}{\partial t}
					\end{equation}
				\item Fuerza generalizada:
					\begin{equation}
						\tcbhighmath[boxrule=1.5pt]{Q_j = \sum_i F_i \frac{\partial r_i}{\partial q_j}}
					\end{equation}

			\end{itemize}
			}

		\item{ \textbf{Ecuaciones de Lagrange}

			Condición: Ligaduras holónomas, $f(r_1, r_2, ..., t) = 0$, entonces:
			\begin{equation}
				\ddt \left( \frac{\partial T}{\partial \dot{q}_j}\right) - \frac{\partial T}{\partial q_j} = Q_j
			\end{equation}

			Condiciones: Ligaduras holónomas Y Fuerzas derivan de un potencial escalar y no depende de la velocidad, $V \neq V(\dot{q}_j)$, entonces:
			\begin{equation}
				\tcbhighmath[boxrule=1.5pt]{\ddt \left( \frac{\partial \Lagr}{\partial \dot{q}_j}\right) - \frac{\partial \Lagr}{\partial q_j} = 0}
			\end{equation}

			tambien:
			\begin{equation}
				Q_j = -\frac{\partial V}{\partial q_j}
			\end{equation}

		}
		\end{itemize}

	\clearpage
	\section{Formulación Hamiltoniana:}

		\begin{itemize}
			\item{ \textbf{Momento generalizado canónico: }
			\begin{equation}
				\tcbhighmath[boxrule=1.5pt]{P_j = \frac{\partial \Lagr}{\partial \dot{q}_j}}
			\end{equation}
			}

			\item{ \textbf{Coordenadas cíclicas: }
			Las coordenadas cíclicas no aparecen explicitamente en el Lagrangiano del sistema
			\begin{equation}
				\Lagr \neq \Lagr(q_c) \Longrightarrow \frac{\partial \Lagr}{\partial q_c} = 0
			\end{equation}
			Por lo tanto:
			\begin{equation}
				\ddt \frac{\partial \Lagr}{\partial \dot{q}_c} = 0 \Rightarrow \ddt P_c = 0 \Longleftrightarrow P_c \equiv cte
			\end{equation}
			}
			\item{ \textbf{Conservación del momento lineal: }

			Si la coordenada $q_j$ es una coordenada de traslación, es decir, $dq_j$ supone un desplazamiento del sistema en la dirección $\hat{n}$ y además es coordenada cíclica:
			\begin{equation}
				P_j \equiv p_{\hat{n}} = cte
			\end{equation}
			}

			\item{ \textbf{Conservación del momento angular: }
			
			Si la coordenada $q_j$ es una coordenada de rotación, es decir, $dq_j$ supone una rotación del sistema respecto a un eje $\hat{n}$ y, además, es coordenada cíclica:
			\begin{equation}
				P_j \equiv L_{\hat{n}} = cte
			\end{equation}
			}

			\item{ \textbf{Conservación de la energía total: }

			Condiciones:
				\begin{itemize}
				\item $V \neq V(\dot{q})$, Las fuerzas derivan de un potencial independiente de las velocidades

				\item Ligaduras independientes del tiempo. $f \neq f(t)$
				\end{itemize}

			Definimos el Hamiltoniano del sistema:

			\begin{equation}
				\tcbhighmath[boxrule=1.5pt]{H =  \sum_j \dot{q}_j P_j - \Lagr}
			\end{equation}

			}

			\item{ \textbf{Teorema: } \label{Teorema.Hamilton}

			Sistema dinámico de s grados de libertad, configuración determinada por las coord. generalizadas $q_j$. Si
			\begin{equation}
				\frac{\partial \Lagr}{\partial t} = 0 \Longrightarrow H = cte
			\end{equation}

			$H = T + V$, si:
			\begin{itemize}
			\item V no es función explicita de $\dot{q}_j$ ni de $t$:
				\begin{equation}
					\frac{\partial V}{\partial \dot{q}_j} = 0 \hspace{5mm} y \hspace{5mm} \frac{\partial V}{\partial t} = 0
				\end{equation}

			\item Las ecs. de transformación $\vec{r}_i = \vec{r}_i(q_j)$ no interviene t:
				\begin{equation}
					\frac{\partial \vec{r}_i}{\partial t} = 0
					\label{Ecs.Trans.Indep.Tiempo}
				\end{equation}

			\item T es función cuadrática homogénea de las velocidades:
				\begin{equation}
					T(k \dot{q}_j) = k^2 T(\dot{q}_j)
					\label{T.Func.Cuad.Hmg}
				\end{equation}
			\end{itemize}

			Condiciones \ref{T.Func.Cuad.Hmg} y \ref{Ecs.Trans.Indep.Tiempo} son equivalentes.
			}

			\item{ \textbf{Ecuaciones de Hamilton: }

			Queremos pasar de $\Lagr \equiv \Lagr(q_j, \dot{q}_j, t)$ a otra función $H \equiv H(q_j, P_j, t)$

			%\begin{equation}
			%	\colorboxed{red}{\dot{q}_j = \frac{\partial H}{\partial P_j}; \hspace{5mm} \dot{P}_j = - \frac{\partial H}{\partial q_j}; \hspace{5mm} \frac{\partial H}{\partial t} = -\frac{\partial \Lagr}{\partial t}}
			%	\label{Ecs.Hamilton}
			%\end{equation}

			\begin{equation}
				\tcbhighmath[boxrule=1.5pt]{\dot{q}_j = \frac{\partial H}{\partial P_j}; \hspace{5mm} \dot{P}_j = - \frac{\partial H}{\partial q_j}; \hspace{5mm} \frac{\partial H}{\partial t} = -\frac{\partial \Lagr}{\partial t}}
				\label{Ecs.Hamilton}
			\end{equation}
			}

			\item{ \textbf{Resolución práctica: }

			\begin{itemize}
				\item Obtención de $\Lagr$

				\item Cálculo de $P_j = \partial \Lagr / \partial q_j$

				\item Obtención de $H =  \sum_j \dot{q}_j P_j - \Lagr$ o bien aplicando el teorema

				\item Cálculo de las ecuaciones de Hamilton (\ref{Ecs.Hamilton})
			\end{itemize}

			}

		\end{itemize}

\end{document}
